% !TEX TS-program = xelatex
% !TEX encoding = UTF-8 Unicode
% !Mode:: "TeX:UTF-8"
\documentclass{resume}
\usepackage{zh_CN-Adobefonts_external} % Simplified Chinese Support using external fonts (./fonts/zh_CN-Adobe/)
%\usepackage{zh_CN-Adobefonts_internal} % Simplified Chinese Support using system fonts
\usepackage{linespacing_fix} % disable extra space before next section
\usepackage{cite}

\begin{document}
\pagenumbering{gobble} % suppress displaying page number

\resumetitle{Yao.Li}

\section{ABOUT ME}
\contactInfo{ShanDong Provience}{Male}{1990.08}{blog: blog.whatsroot.xyz}
\contactInfo{Tel: 173-0548-7880}{EMAIL: shanhu21@sina.cn}{}{}
\contactInfo{SKILL: c/c++;python;Qt}{Position Applied: Software develop}{Excepted Work Area: HanZhou}{}
 
\section{EDUCATION}
\datedsubsection{\textbf{University of Chinese Academy of Sciences} \quad \textit{particle physics and nuclear physics} \quad \textit{Master of Science}}{2014.9 - 2017.6}
\datedsubsection{\textbf{China University of Mining and Technology} \qquad \textit{applied physics} \qquad \qquad \qquad \quad \textit{bachelor of science}}{2009.9 - 2013.6}

\section{WORK EXPERIENCE}

\datedsubsection{\textbf{Dahua Technology} \quad \textit{Embedded Linux Software Engineer} \quad \textit{video and sound}}{2019.5 - 今}
\begin{itemize}[parsep=0.2ex]
  \item \textbf{海思平台音视频系统开发}: \newline
    动力环境监控设备,负责音视频开发,设计实现了码流无编解码的OSD叠加方案。
  \item \textbf{rk新平台音视频开发}: \newline
    基于linux音频子系统ALSA的音频开发。包括适配codec内核驱动,抽象音频alsa接口,添加alsa插件,修改alsa源码增加滤波等。
    工作内容为封装alsa接口,向上提供简单易用的接口。
  \item \textbf{mstar新平台音视频开发}:\newline
    新平台,适配新的SDK框架。
  \item \textbf{内部开源工具}:\newline
    python开发命令行工具,解析xml文件生成配置文件,根据配置文件实现自动从svn下载库代码,自动解析编译命令,自动拷贝生成产物,
    后期根据需求增加了自动修改xml文件版本号,自动发起三方编译,自动从三方服务器下载编译产物等多种功能。
\end{itemize}

\datedsubsection{\textbf{Amaris(ShangHai)Consulting} \quad \textit{Embedded Software Engineer} \quad \textit{Automated testing for embedded system}}{2019.1 - 2019.4}
\begin{itemize}[parsep=0.2ex]
  \item \textbf{电动汽车VCU开发}: 自动化测试程序
\end{itemize}

\datedsubsection{\textbf{Shenzhen Klclear Technology} \quad \textit{Embedded Linux Software Engineer} \quad \textit{电池管理系统开发}}{2017.7 - 2019.1}
\begin{itemize}[parsep=0.2ex]
    \item \textbf{电动汽车BMS开发}: \newline
    基于AUTOSAR的新平台电池管理系统(BMS)开发,负责诊断系统开发:采集处理电池单体的电压,电流,温度信息,设计实现诊断策略,处理异常信息诊断,
    处理电压温度电流异常时的状态保存及信息上报, 实现功能安全\emph{ISO26262}要求
  \item \textbf{诊断系统客户端开发}: \newline
    Qt开发的上位机工具,根据标准文档\emph{ISO14229}原文要求实现UDS诊断功能。
    开发期间逐页研读\emph{ISO14229}英文原文,独立设计完成了模块化的诊断功能,支持多语言,支持多种CAN通讯设备,为支持不同厂家的不同协议,
    可根据加载的配置文件改变客户端外观,改变诊断顺序。
  \item \textbf{其他工作}: \newline
    单片机上CAN,UART,SPI驱动开发
\end{itemize}

\section{个人总结}
\begin{itemize}[parsep=0.2ex]
  \item \textbf{对新技术充满热情}\newline
    喜欢钻研新技术,比如开发ALSA音频时研究阅读内核及alsa库的源代码,开发UDS诊断时阅读ISO文档原文深入了解协议的内容,不断尝试优化完成新技术开发。
  \item \textbf{善于自我总结与提高}\newline
    在解决掉复杂的问题或者系统学习过某个内容后都会重新整理记录,
比如在阅读linux音频子系统源代码时对ALSA的跟踪分析(https://blog.whatsroot.xyz/2020/08/11/alsa\_snd\_open-analysis-6/)就是在一点点阅读理解源代码后做的总结,
这些内容都是无论中文还是英文搜索引擎上都没有的首次详细深入的分析过程。
  \item \textbf{乐观自信,坚持阅读}\newline
    硕士期间是暗物质探测卫星(悟空号)正式成员,相关成果已在\emph{Nature}发表,在最前沿的科研经历锻炼了对未知难题的处理方法与处理心态,
    了解自身还有哪些欠缺的内容,了解工作中还需要补充哪些内容,了解如何阅读以及阅读什么书籍来满足工作需要。
\end{itemize}

\end{document}
