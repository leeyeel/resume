% !TEX TS-program = xelatex
% !TEX encoding = UTF-8 Unicode
% !Mode:: "TeX:UTF-8"
\documentclass{resume}
\usepackage{zh_CN-Adobefonts_external} % Simplified Chinese Support using external fonts (./fonts/zh_CN-Adobe/)
%\usepackage{zh_CN-Adobefonts_internal} % Simplified Chinese Support using system fonts
\usepackage{linespacing_fix} % disable extra space before next section
\usepackage{cite}

\begin{document}
\pagenumbering{gobble} % suppress displaying page number

\resumetitle{李瑶}

\section{个人信息}
\contactInfo{籍贯:山东}{性别: 男}{出生日期: 1990.08}{个人博客: blog.whatsroot.xyz}
\contactInfo{电话: 17305487880}{电子邮箱: shanhu21@sina.cn}{}{}
\contactInfo{语言: c/c++;python;Qt}{期望职位:音视频SDK开发}{期望地点:杭州}{}
 
\section{教育背景}
\datedsubsection{\textbf{中国科学院大学} \quad \textit{粒子物理与原子核物理} \quad \textit{理学硕士}}{2014.9 - 2017.6}
\datedsubsection{\textbf{中国矿业大学} \qquad \textit{应用物理} \qquad \qquad \qquad \quad \textit{理学学士}}{2009.9 - 2013.6}

\section{工作及项目经历}

\datedsubsection{\textbf{浙江大华股份有限公司} \quad \textit{嵌入式linux软件开发} \quad \textit{音视频系统}}{2019.5 - 今}
\begin{itemize}[parsep=0.2ex]
  \item \textbf{海思平台音视频系统开发}: \newline
    动力环境监控设备,负责音视频开发,实现了码流无编解码的OSD叠加方案,难点在于多通道码流互不影响,同时需要快速响应
  \item \textbf{rk新平台室内机音视频开发}: \newline
    基于linux音频子系统ALSA的音频开发。包括适配codec内核驱动,抽象音频alsa接口,添加alsa插件,修改alsa源码增加高通滤波等。
    此项目的难点在于平台无SDK,需要自己抽象接口。
  \item \textbf{sigmastar新平台音视频系统开发}:\newline
    新平台,新SDK,需要重新调整框架
  \item \textbf{内部开源工具}:\newline
    因为在服务器上编译各个库不方便,还设计到拷贝生成产物,对库进行代码静态扫描,从三方服务器下载静态库等多种操作,
    使用python开发命令行工具,针对不同场景实现一键操作,把计算机能做的事情还给计算机操作。
\end{itemize}

\datedsubsection{\textbf{AMARIS摩芮思科技咨询有限公司} \quad \textit{嵌入式软件} \quad \textit{自动化测试}}{2019.1 - 2019.4}
\begin{itemize}[parsep=0.2ex]
  \item \textbf{电动汽车VCU开发}: 自动化测试程序
\end{itemize}

\datedsubsection{\textbf{深圳科列技术股份有限公司} \quad \textit{嵌入式软件} \quad \textit{电池管理系统开发}}{2017.7 - 2019.1}
\begin{itemize}[parsep=0.2ex]
    \item \textbf{电动汽车BMS开发}: \newline
    基于AUTOSAR的新平台电池管理系统(BMS)开发,负责诊断系统开发:几百个电池单体的电压,电流,温度异常信息诊断处理,信息传输校验等。
    难点是需要复合功能安全\emph{ISO26262}需求,需要逐项比对ISO文档需求。
  \item \textbf{诊断系统客户端开发}: \newline
    完全独立设计开发的诊断系统PC客户端,使用Qt开发,逐页研读\emph{ISO14229}英文原版标准,支持多语言,支持多种CAN通讯设备,为支持不同厂家的不同协议,
    实现了根据加载的配置文件改变客户端外观,改变协议内容。
  \item \textbf{其他工作}: \newline
    单片机CAN,UART,SPI驱动开发
\end{itemize}

\section{个人总结}
\begin{itemize}[parsep=0.2ex]
  \item \textbf{有成熟的科研基础,可探索新技术}\newline
    硕士期间为暗物质探测卫星(悟空号)正式成员,负责塑闪探测器的数据重建分析,相关成果已在\emph{Nature}发表,
    对新事物有浓厚的兴趣及成熟研究方法。
  \item \textbf{善于自我总结与提高}\newline
  个人博客中记录内容全部为原创,且绝大部分都是网络上绝无仅有的教程或总结。
  比如在阅读linux音频子系统源代码时对ALSA的跟踪分析(https://blog.whatsroot.xyz/2020/08/11/alsa\_snd\_open-analysis-6/),无论是中文网络还是英文网络都找不到第二份相关分析。
  \item \textbf{可根据英文论文或标准实现代码}\newline
  博客中有对ISO14229原文难点部分的总结:https://blog.whatsroot.xyz/2019/03/02/UDS-DTC-introduction/
  \item \textbf{从不间断阅读}\newline
  有坚持读书的习惯,从物理专业到目前写代码,都是基于平时不间断的阅读,读书是使用google或者请教别人等都无法替代的最能深入的学习方式。
\end{itemize}

\end{document}
